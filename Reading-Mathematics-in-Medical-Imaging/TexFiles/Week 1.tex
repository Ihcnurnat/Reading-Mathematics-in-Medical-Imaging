\section{Week 1. The Space of Lines in the Plane}

Instead of using a Cartesian coordinate system, we adopt a “point normal” parameterization for an arbitrary line in the plane, introduced in \textbf{Section 1.2}.
Recall  a line in the plane is a set of points that satisfies 
and equation of the form
\[
a x + by  = c,
\]
where $a^2 + b^2 \neq 0$.
We can rewrite this equation as 
\[
\frac{a}{\sqrt{a^2 + b^2}} x + \frac{b}{\sqrt{a^2 + b^2}} y = \frac{c}{\sqrt{a^2 + b^2}},
\]
which represents the same line. 
The coefficients $(\frac{a}{\sqrt{a^2 + b^2}}, \frac{a}{\sqrt{a^2 + b^2}})$ define a point $\omega$ on the unit circle $S^1$ and $\frac{c}{\sqrt{a^2 + b^2}}$ can be any number.

This motivates us to parameterize a line in the plane by a pair of a unit vector $\omega(\theta) = (\cos \theta, \sin \theta)$ and a real number $t$.
We call such a line $l_{t,\theta}:=l_{t, \omega(\theta)}$ and it is the set of all points $(x,y)$ in $\mathbb{R}^2$ such that 
\[
x\cos \theta  + y\sin \theta = t, \quad \theta \in [0, 2\pi), \ t \in \mathbb{R}.
\]
Note that $\omega(\theta)$ is perpendicular to $l_{t, \omega}$.
We can verify $l_{t, \omega} = l_{-t, -\omega}$.
In fact, the pair $(t,\omega)$ specifies an \textit{oriented line}. Define a unit vector 
\[\hat{\omega}=(-\sin \theta,\cos \theta)\] where recall $\omega=(\cos \theta,\sin \theta)$. 
With such notations, we have a bijection as follows.


\begin{Proposition}{(Proposition 1.2.1.)}%{Proposition 1.2.1.}
\textit{There is a bijection between} 
% https://q.uiver.app/#q=WzAsMixbMCwwLCJcXHsodCxcXG9tZWdhKSB+XFxtaWR+IHRcXGluIFxcbWF0aGJiIFIsfiBcXG9tZWdhXFxpblxcbWF0aGJiIFNcXH0iXSxbMSwwLCJcXHtcXHRleHR7IG9yaWVudGVkIGxpbmVzIGluIHRoZSBwbGFuZSB9XFx9Il0sWzAsMSwiXFx0ZXh0e2JpamVjdGl2ZX0iLDAseyJzdHlsZSI6eyJ0YWlsIjp7Im5hbWUiOiJob29rIiwic2lkZSI6InRvcCJ9LCJoZWFkIjp7Im5hbWUiOiJlcGkifX19XV0=
\[\begin{tikzcd}
	{\{(t,\omega) ~\mid~ t\in \mathbb R,~ \omega\in\mathbb S\}} & {\{\text{oriented lines in the plane}\}}
	\arrow["{\text{bijective}}", hook, two heads, from=1-1, to=1-2]
\end{tikzcd}.\]
\end{Proposition}

\begin{Exercise}{(Exercise 1.2.1.)}
\textit{Show that $l_{t,\omega}$ is given parametrically as the set of points
\[
l_{t, \omega} = \{t \omega + s \hat{\omega}: s \in (-\infty, \infty)\}.
\]}
\end{Exercise}
\begin{anw}
Simplify the expression we'll get exactly the same expression. It's a good way to visualise \textit{affine parameter} $t$. In this equation, it represents precisely the length from origin to the line where the direction is specified by vector $\omega$.
\end{anw}

\begin{Exercise}{(Exercise 1.2.2.)}
\textit{Show that $\hat{\omega}(\theta) = \partial_\theta \omega(\theta)$}
\end{Exercise}
\begin{anw}
Differentiate component-wise will work.
\end{anw}