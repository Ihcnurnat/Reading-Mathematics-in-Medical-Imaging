\section{Week 2. A Basic Model for Tomography}
\subsection{Beer's Law}

%Key concepts: attenuation coefficient, Beer's law
We start our exploration of medical imaging, with a mathematical representation of the measurement process used in X-ray tomography. 
The modeling process commences with a detailed quantitative analysis of how X-rays interact with matter, a phenomenon described by Beer’s law, see \textbf{Section 3.1}.
\begin{df} 
In X-ray tomography, we are interested in detecting objects using real-valued funcion defined on $\mathbb R^3$. And we define this function as \textit{attenuation coefficient}.
\end{df}
The attenuation coefficient quantifies the tendency of an object to absorb or scatter X-ray of a given energy. For example, bone has a much higher attenuation coefficient than soft tissue.


\begin{df}
For radiologists, attenuation coefficient is compared to the attenuation coefficient of water and we define it in terms of a dimensionless quantity, for which we call \textit{Hounsfield unit}. And we define the normalized attenuation coefficient in Hounsfield units as 
\[H_{\text{tissue}}:=\frac{\mu_{\text{tissue}}-\mu_{\text{water}}}{\mu_{\text{water}}}\times 1000.\]  
\end{df}

In practice of measurement, it's difficult to distinguish points where a function is nonzero from points that are "arbitrarily closed" to such points. This motivates us to make a definition as follows:

\begin{df} Let $f$ be a function defined on $\mathbb R^n$, a point $\mathbf{x}\in\mathbb R^n$ belongs to \textit{support} of $f$ if there's a sequence of points $\langle \mathbf{x}_n\rangle$ such that 
\[f(\mathbf{x}_n)\neq 0,~ \lim_{n\to\infty}\mathbf{x}_n=\mathbf{x}.\]
And we denote the set of all such points as $\operatorname{supp}(f)$.
\end{df}

Usually, X-ray beam is described by a vector valued function $\mathbf{I}(\mathbf x)$. The direction of $\mathbf I$ at $\mathbf x$ is the flux at $\mathbf x$ and its magnitude, and we denote the intensity of the beam as \[I(\mathbf x)=\|\mathbf I(\mathbf x)\|.\]

The model discussed in the textbook for the interaction of X-rays with matter is phrased in terms of the continuum model and rests on three basic assumptions:
\begin{itemize}
    \item[(1)] No refraction or diffraction, as X-rays have very high energies.
    \item[(2)] Monochromatic, waves of X-ray beam are of the same frequency.
    \item[(3)] \textbf{Beer's Law:} the intensity, $I$ of the X-ray beams, satisfies
    \[\frac{d I}{ds}=-\mu(x)I\] where $s$ is the arc-length along the straight-line trajectory of the X-ray beam.
\end{itemize} 

\begin{Definition} See Example 3.1.5. \cite{epstein2007introduction} for \textit{isotropic}.
\end{Definition}

Figure 3.3 gives an example of a failure where we couldn't distinguish objects. However, figure 3.4 gives us a general principle: to distinguish more arrangements of objects we have to make measurements from more directions.

\subsection{Shepp-Logan Phantom and its Line Integrals}

Understand Figure 3.7 and solve the exercises.

\begin{Exercise} 
    Think about Exercise 3.2.3.
\end{Exercise}

\begin{Exercise}{Exercise 3.2.3.} \textbf{Answer:} White bands corresponding to the skull of the highest attenuation. The white band is curved like a wave because the length varies from given different angle $\theta$.
\end{Exercise}


%\subsection{Chapter 3.3} 
Note: You can skip \textbf{Section 3.3} currently!

\subsection{Radon Transform} 
%Key concepts: Radon transform and its properties, back-projection

\begin{Exercise}  
Think about how to solve 3.4.3.
\end{Exercise}

\begin{Definition} For simplicity we assume $f$ is a function defined on the plane that is continuous with bounded support. The integral of $f$ along the line $l_{t,\omega}$ is denoted by 
\[\mathscr R f(t,\omega) = \int_{-\infty}^{\infty} f(s\hat{\omega}+t\omega)ds.\]
The collection of integrals of $f$ along the lines in he plane defines a function on $\mathbb R\times \mathbb S^1$, called the \textit{Radon transform} of $f$.
\end{Definition}

\begin{remark} We have following remarks, according to \textbf{Section 3.4.}
\begin{itemize}
    \item[(1)] Properties of Radon transform: linear, monotone, and $Rf$ is an even function.
    \item[(2)] Radon transform cannot distinguish functions which differ only on a set of measure zero, which is a feature common to any measurement process defined by integrals. 
    \item[(3)] The Radon transform can be defined for a function $f$ whose restriciotn to each line is locally integrable and 
    \[
    \int_{-\infty}^{\infty} |f(t\omega + s \hat{\omega}) d s | < \infty, \quad \text{for all } (t, \omega) \in \mathbb{R} \times S^1.
    \]
    Functions that satisfy this are in the \textit{natural domain} of the Radon transform.
\end{itemize}
\end{remark}

\subsection{Integrable Functions}
We briefly recall some definitions in \cite[Chap 2 \& 3]{stein2009real}. 
For a real-valued measurable function $f$ on $\Rn$, we say that $f$ is \textbf{Lebesgue integrable} if the non-negative measurable function $|f(x)|$ is integrable, that is, its Lebesgue integral $\int_{\Rn}  |f(x)| d x < +\infty$. 
In fact, the integrable functions form a vector space, where we can define the \textbf{norm} of $f$ as
\[
\|f\| \coloneqq \|f\|_{L^1(\Rn)} \coloneqq \int_{\Rn} |f(x)| dx.
\]
The collection of all integrable functions with the above norm gives a (somewhat imprecise) definition of the space $L^1(\Rn)$. For more details, see \cite[Section 2.2]{stein2009real}.

We say $f$ is \textbf{locally integrable}, if for every ball $B$ in $\Rn$, the function $f(x) \chi_B(x)$ is integrable, where $\chi_B(x)$ is the characteristic function on $B$.
We denote by $L^1_\mathrm{loc}(\Rn)$ the space of all locally integrable functions.

The space of square integrable functions on $\Rn$ is denoted by $L^2(\Rn)$.
It consists of all mensurable functions $f$ that satisfy 
\[
\int_{\Rn} |f(x)|^2 d x < +\infty.
\]
The resulting $L^2(\Rn)$-norm is defined by 
\[
\|f\|_{L^2
(\Rn)} \coloneqq (\int_{\Rn} |f(x)|^2 d x)^{1/2}.
\]