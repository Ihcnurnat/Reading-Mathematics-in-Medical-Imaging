\section{Week 7.}

\subsection{Tools from Appendix B: Basic Analysis}

\begin{theorem}[Theorem B.8.1 (Fubini's Theorem)]
    Let $f$ be a function defined on $\mathbb R^n$ and let $n=k+l$ for positive integers $k$ and $l$. If either of the iterated integrals 
    \[\int_{\mathbb R^k}\int_{\mathbb R^l}\vert f(\mathbf w,\mathbf y)\vert \,d\mathbf w\,d\mathbf y~\text{ or }~\int_{\mathbb R^l}\int_{\mathbb R^k}\vert f(\mathbf w,\mathbf y)\vert \,d\mathbf y\,d\mathbf w\]is finite, then the other is as well. In this case $f$ is integrable over $\mathbb R^n$ and 
    \[\int_{\mathbb R^k}\int_{\mathbb R^l}\vert f(\mathbf w,\mathbf y)\vert \,d\mathbf w\,d\mathbf y=\int_{\mathbb R^n}f(\mathbf x)\,d\mathbf x=\int_{\mathbb R^l}\int_{\mathbb R^k}\vert f(\mathbf w,\mathbf y)\vert \,d\mathbf y\,d\mathbf w.\]
\end{theorem}

\begin{remark}
    Informally speaking, the order of the integrations can be interchanged under the assumption that $f$ is \textit{absolutely} integrable. Because there are examples of functions defined on $\mathbb R^2$ so that both iterated integrals
    \[\int\int f(x,y)dxdy,~\int\int f(x,y)dydx\] exist but are not equal, with $f$ not integrable on $\mathbb R^2$.
\end{remark}

\subsection{The Heisenberg Uncertainty Principle}

Here we study relationships between the $\operatorname{supp} f$ and $\operatorname{supp}\hat{f}$. The simplest such result states that if a function has bounded support, then its Fourier transform cannot.

\begin{proposition}[Proposition 4.4.1.]
    Suppose $\operatorname{supp}f$ is contained in the bounded interval $(-R,R)$. If $\hat{f}$ also has bounded support then $f\equiv 0$.
\end{proposition}

\subsection{The Fourier Transformation for Functions of Several Variables}

We make some notation conventions that will be useful for multi-variable functions. We often use lowercase Roman bold letter to denote 
\[\mathbf x=(x_1,...,x_n)\in\mathbb R^n.\]
And it's customary to use lowercase Greek bold letters for points on the Fourier transform space such as 
\[\mathbf{\xi}=(\xi_1,...,\xi_n).\]

Again, we start by defining for functions from $L^1(\mathbb R^n)$.
\begin{definition}[Definition 4.5.1.]
    If $f$ belongs to $L^1(\mathbb R^n)$, then the Fourier transform, $\hat f$ of $f$, is defined by 
    \[\hat f(\mathbf{\xi})=\int_{\mathbb R^n} f(\mathbf x)e^{-i\langle \mathbf{\xi},\mathbf x\rangle}\,d\mathbf x ~\text{ for }~ \mathbf{\xi}\in\mathbb R^n.\]
\end{definition}
Furthermore, we can express this integral as 
\[\int_{-\infty}^{\infty}\cdots\int_{-\infty}^{\infty}f(x_1,...,x_n)e^{-ix_1\xi_1}\,dx_1\cdots e^{-ix_n\xi_n}\,dx_n.\]
While $f$ is assumed to be absolutely integrable, then Fubini's theorem ensures that we can interchange the order of integration. 

\begin{theorem}[Theorem 4.5.1 (Fourier Inversion Formula)]
    Suppose that $f$ is an $L^1$-function defined on $\mathbb R^n$. If $\hat f$ also belongs to $L^1(\mathbb R^n)$, then 
    \[f(\mathbf x)=\frac{1}{[2\pi]^n}\int_{\mathbb R^n} \hat f(\mathbf{\xi})e^{i\mathbf x\cdot \mathbf{\xi}}\,d\mathbf{\xi}.\]
\end{theorem}