\section{Week 4 Dominated Convergence Theorem}

In this week we read some definitions and results in Stein's textbook \cite{stein2009real}.

\subsection{Measurable Functions}
\begin{Theorem}[{\cite[Chapter 1, Theorem 1.4 \& Chapter 2, Theorem 1.13]{stein2009real}}]
    Every open subset $\mathcal O$ of $\mathbb R^d$, $d \geq 1$, can be written as a countable union of \textit{almost disjoint} closed cubes.
\end{Theorem}

Exterior measure $m_{\ast}$, which attempts to describe the volume of a set $E$ by approximating it from the outside, assigns to any subset of $\mathbb R^d$ a first notion of size.

Properties of Exterior Measure

\subsection{Convergence Theorem}
\begin{Theorem}[{\cite[Chapter 2, Theorem 1.13]{stein2009real}}]\label{thm_dominated}
    Suppose $\{f_n\}$ is a sequence of measurable functions such that $f_n(x) \to f(x)$ a.e. $x$, as n tends to infinity. If $\vert f_n(x)\vert \leq g(x)$, where $g$ is absolutely integrable, then
    \[\int \vert f_n-f\vert \to 0 ~\text{ as }~ n\to \infty,\]
    and consequently
    \[\int f_n\to \int f ~\text{ as }~ n\to \infty.\]
\end{Theorem}



In this case, we can first integrate and then take the limit, which will give us $\int f$.


\subsection{Some Notations}
The limit inferior and limit superior of a sequence ${x_n}$ are defined by
\[
\liminf_{n \rightarrow \infty} x_n \coloneqq \lim_{n \rightarrow \infty}(\inf_{m \geq n} x_m),
\qquad 
\limsup_{n \rightarrow \infty} x_n \coloneqq \lim_{n \rightarrow \infty}(\sup_{m \geq n} x_m).
\]
If $\liminf_{n \rightarrow \infty} x_n$ exists, then it is the largest real number $a$ such that for any $\epsilon > 0$, there exists an integer $N$ satisfying $x_n > a - \epsilon$ for any $n > N$.
Only finitely many elements of the sequence are less than $a - \epsilon$.
Similarly, if $\limsup_{n \rightarrow \infty} x_n$ exists, then it is the smallest real number $b$ such that for any $\epsilon > 0$, there exists an integer $N$ satisfying $x_n < b + \epsilon$ for any $n > N$.
Only finitely many elements of the sequence are greater than $b + \epsilon$.