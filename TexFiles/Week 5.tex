\section{Week 5. Properties of Fourier Transform}
As is discussed below \textbf{Remark 4.2.2,} the operation performed to recover $f$ from $\hat{f}$ is almost the same as the operation performed to obtain $\hat{f}$ from $f$, if we compare the Fourier transform with the inverse Fourier transform. 
Indeed, if we define
\[
f_r(x) \coloneqq f(-x),
\]
then we have
\[
\mathcal{F}^{-1}(f) = \frac{1}{2\pi} \int_{-\infty}^\infty {f}(\xi) e^{i x \cdot \xi} d \xi
 = \frac{1}{2\pi} \hat{f}(-x) = \frac{1}{2\pi} \mathcal{F}(f_r).
\]
This illustrates some ``symmetry" between Fourier transform and its inverse and accounts for many of the Fourier transform's properties. 

\subsection{Regularity and Decay}

It is a general principle that the regularity properties of a function $f$ on $\Rn$ are reflected in the decay properties of its Fourier transform $\hat{f}$ and similarly, the regularity of $\hat{f}$ is a reflection of the decay properties of $f$.

\begin{Theorem}[Theorem 4.4.2 \textbf{The Riemann-Lebesgue Lemma}]
    \textit{If $f$ is an $L^1$-function, then its Fourier transform $\hat{f}$ is a continuous function that goes to zero at infinity. That is, for $\eta\in\mathbb R$, \begin{align*} \lim_{\xi\to \eta} \hat{f}(\xi)=\hat{f}(\eta) ~\text{ and }~ \lim_{\xi\to\pm\infty} \hat{f}(\xi)=0.\end{align*}}
\end{Theorem}

\begin{proof}
First, we prove $\hat{f}(\xi)$ is continuously (actually it is uniformly continuous). 
For this purpose, let $h> 0$ be small and we compute
\begin{align*}
    |\hat{f}(\xi + h) - \hat{f}(\xi)| = 
\end{align*}
\end{proof}

\begin{Definition}[Definition 4.2.2.]
For $k\in\mathbb N\cup\{0\}$, the set of functions on $\mathbb R$ with $k$ continuous derivatives is denoted by $\mathscr C(\mathbb R)$. The set of infinitely differentiable functions is denoted by $\mathscr C^{\infty}(\mathbb R)$.
    
\end{Definition}

\begin{Definition}[Definition 4.2.3.]
    A function, $f$, defined on $\mathbb R^d$, decays like $\lVert\mathbf{x}\rVert^{-a}$ if there are constant $C$ and $R$ so that 
    \[\mid f(\mathbf{x})\mid \leq \frac{C}{\lVert \mathbf a\rVert } ~\text{ for }~ \lVert \mathbf x\rVert > R.\]
\end{Definition}
And we use the notation "$f=\mathcal O(\lVert \mathbf x\rVert^{-a})$ as $\lVert\mathbf x\rVert$ tends to infinity."

\begin{Question}
    Notation for $f^{[j]}$? \textbf{Answer:} Here $f^{[j]}(x)$ denotes the $j$th derivation of $f(x)$. % Got it, thanks!
\end{Question}

\subsection{Quantitative Measures of Regularity and Decay}
Recall the integration by parts formula: for differentiable function $f$ and $g$ on the interval $[a,b]$, we have 
\[
\int_a^b f'(x) g(x) d x = f(x) g(x)\bigg\rvert_a^b - \int_a^b f(x) g'(x) d x.
\]
To use integration by parts in Fourier analysis, we need to consider this formula when $a = -\infty$ and $b = + \infty$. 
For our purpose, if we assume $fg, f'g, fg'$ are absolutely integrable, then we have 
\[
\lim_{x \rightarrow \pm \infty} fg(x) = 0,
\]
and therefore we have
\begin{align}\label{eq_intpart}
    \int_{-\infty}^{\infty} f'(x) g(x) d x = - \int_{-\infty}^{\infty} f(x) g'(x) d x.
\end{align}
Suppose $f$ has $j$ integrable derivatives, for $j \geq 1$.
Then for any $\xi \neq 0$, we can use (\ref{eq_intpart}) to obtain a formula that relates $\mathcal{F}(f)$ with $\mathcal{F}(f^{[j]})$:
\begin{align*}
    \hat{f}(\xi) = \int_{-\infty}^{\infty} f(x) e^{-i x\xi}d x 
    = \int_{-\infty}^{\infty} f'(x) \frac{e^{-i x\xi}}{i \xi}d x 
    = \ldots 
    = \int_{-\infty}^{\infty} f^{[j]}(x) \frac{e^{-i x\xi}}{(i \xi)^j}d x. 
\end{align*}
Note that the last equality can be regarded as $\frac{1}{(i \xi)^j}$ multiplied by the Fourier transform of $f^{[j]}(x)$. 
Thus, we conclude that
\[
\mathcal{F}(f) = \frac{1}{(i \xi)^j} \mathcal{F}(f^{[j]})
\]
when $f$ has $j$ integrable derivatives. 

\begin{Proposition}[Proposition 4.2.1.]
\textit{Let $j$ be a positive integer. If $f$ has $j$ integrable derivatives, then there is a constant $C$ so $\hat{f}$ satisfies the estimate \[\vert \hat{f}(\xi)\vert \leq \frac{C}{(1+\mid \xi\mid)^j}.\] Moreover, for $1\leq l\leq j$, the Fourier transform of $f^{[l]}$ is given by \[\widehat{f^{[l]}}(\xi)=(i\xi)^l \hat{f}(\xi).\] The rate of decay in $\hat{f}$ is also reflected in the smoothness of $f$.}
\end{Proposition}

\begin{Example}[Example 4.2.2. (\textbf{Sinc Function})]
    See a post \href{https://math.stackexchange.com/questions/390810/improper-integral-sinx-x-converges-absolutely-conditionally-or-diverges}{HERE} computing the last integral.
\end{Example}

\begin{Example}[Example 4.2.3.]
    
\end{Example}

\subsection{The Parseval Formula}
\begin{Definition}[Definition 4.2.4.]
A complex-valued function $f$, defined on $\mathbb R^n$, is $L^2$ or \textit{square integrable} if 
\[\lVert f\rVert^2 _{L^2} = \int_{\mathbb R^n} \mid f(\mathbf x)\mid ^2d\mathbf{x} <\infty.\]
\end{Definition}
Denote the set of all such functions, with norm defined by $\lVert \cdot\rVert_{L^2}$, by $L^2(\mathbb R^n)$. And with such norm $L^2(\mathbb R^n)$ is a complete, normed vector space. 
Here the norm on $L^2(\mathbb R^n)$ is defined by an inner product,
\[\langle f,g\rangle_{L^2} = \int_{\mathbb R^n} f(\mathbf x)\overline{g(\mathbf x)}d\mathbf x.\]
And this inner product satisfies Cauchy-Schwarz inequality.
\begin{Proposition}[Proposition 4.2.4.]\label{Proposition 4.2.4.}
\textit{If $f,g\in L^2(\mathbb R^n)$, then \[\mid\langle f,g\rangle_{L^2}\mid \leq \lVert f\rVert_{L^2}\lVert g\rVert_{L^2}.\]}
\end{Proposition}

The relationship between absolutely integrable functions and square integrable functions is complicated in the sense that neither of them contains the other one. Some pathological examples are provided below.
\begin{Example}[Example 4.2.7.]
The function \[f(x)=(1+\mid x\mid)^{-\frac{3}{4}}\] is not absolutely integrable, but it is square integrable. On the other hand, the funciton \[g(x)=\frac{\chi_{[-1,1]}(x)}{\sqrt{\mid x\mid}}\] is absolutely integralble but not square integrable.

See a post \href{https://math.stackexchange.com/questions/18395/what-functions-or-classes-of-functions-are-l1-but-not-l2}{HERE}.

\begin{proof}
    It's not absolutely integrable because 
    \begin{align*}
        \int_{\mathbb R} \vert f(x)\vert \, dx \geq \int_{0 < x < \infty} \vert f(x)\vert \,dx=\int_{0 < x < \infty} \left(\frac{1}{(1+x)^{3/4}}\right)\ ,dx = \infty.
    \end{align*}
    While we can compute $L^2$-norm as 
    \begin{align*}
    \int_{\mathbb R} f(x) =& \int_{\mathbb R}\vert \left( 1+ \vert x\vert \right)^{-\frac{3}{4}}\vert^2
    = \int_{-\infty}^0 (1-x)^{-\frac{3}{2}} \,dx + \cdots
    = 2+2 =4.
\end{align*} 


\end{proof}


\end{Example}



\begin{exercise}[Exercise 4.2.9.]
Let $f$ be an $L^1$-function. Show that $\hat{f}$ is a continuous function. Extra credit: Show that $\hat{f}$ is uniformly continuous on the whole real line.
\end{exercise}